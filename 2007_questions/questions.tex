\documentclass[journal,12pt,twocolumn]{IEEEtran}

\usepackage{setspace}
\usepackage{gensymb}
\singlespacing
\usepackage[cmex10]{amsmath}

\usepackage{amsthm}

\usepackage{mathrsfs}
\usepackage{txfonts}
\usepackage{stfloats}
\usepackage{bm}
\usepackage{cite}
\usepackage{cases}
\usepackage{subfig}

\usepackage{longtable}
\usepackage{multirow}

\usepackage{enumitem}
\usepackage{mathtools}
\usepackage{steinmetz}
\usepackage{tikz}
\usepackage{circuitikz}
\usepackage{verbatim}
\usepackage{tfrupee}
\usepackage[breaklinks=true]{hyperref}
\usepackage{graphicx}
\usepackage{tkz-euclide}

\usetikzlibrary{calc,math}
\usepackage{listings}
    \usepackage{color}                                            %%
    \usepackage{array}                                            %%
    \usepackage{longtable}                                        %%
    \usepackage{calc}                                             %%
    \usepackage{multirow}                                         %%
    \usepackage{hhline}                                           %%
    \usepackage{ifthen}                                           %%
    \usepackage{lscape}     
\usepackage{multicol}
\usepackage{chngcntr}

\DeclareMathOperator*{\Res}{Res}

\renewcommand\thesection{\arabic{section}}
\renewcommand\thesubsection{\thesection.\arabic{subsection}}
\renewcommand\thesubsubsection{\thesubsection.\arabic{subsubsection}}

\renewcommand\thesectiondis{\arabic{section}}
\renewcommand\thesubsectiondis{\thesectiondis.\arabic{subsection}}
\renewcommand\thesubsubsectiondis{\thesubsectiondis.\arabic{subsubsection}}


\hyphenation{op-tical net-works semi-conduc-tor}
\def\inputGnumericTable{}                                 %%

\lstset{
%language=C,
frame=single, 
breaklines=true,
columns=fullflexible
}
\begin{document}


\newtheorem{theorem}{Theorem}[section]
\newtheorem{problem}{Problem}
\newtheorem{proposition}{Proposition}[section]
\newtheorem{lemma}{Lemma}[section]
\newtheorem{corollary}[theorem]{Corollary}
\newtheorem{example}{Example}[section]
\newtheorem{definition}[problem]{Definition}

\newcommand{\BEQA}{\begin{eqnarray}}
\newcommand{\EEQA}{\end{eqnarray}}
\newcommand{\define}{\stackrel{\triangle}{=}}
\bibliographystyle{IEEEtran}
\raggedbottom
\setlength{\parindent}{0pt}
\providecommand{\mbf}{\mathbf}
\providecommand{\pr}[1]{\ensuremath{\Pr\left(#1\right)}}
\providecommand{\qfunc}[1]{\ensuremath{Q\left(#1\right)}}
\providecommand{\sbrak}[1]{\ensuremath{{}\left[#1\right]}}
\providecommand{\lsbrak}[1]{\ensuremath{{}\left[#1\right.}}
\providecommand{\rsbrak}[1]{\ensuremath{{}\left.#1\right]}}
\providecommand{\brak}[1]{\ensuremath{\left(#1\right)}}
\providecommand{\lbrak}[1]{\ensuremath{\left(#1\right.}}
\providecommand{\rbrak}[1]{\ensuremath{\left.#1\right)}}
\providecommand{\cbrak}[1]{\ensuremath{\left\{#1\right\}}}
\providecommand{\lcbrak}[1]{\ensuremath{\left\{#1\right.}}
\providecommand{\rcbrak}[1]{\ensuremath{\left.#1\right\}}}
\theoremstyle{remark}
\newtheorem{rem}{Remark}
\newcommand{\sgn}{\mathop{\mathrm{sgn}}}
\providecommand{\abs}[1]{\left\vert#1\right\vert}
\providecommand{\res}[1]{\Res\displaylimits_{#1}} 
\providecommand{\norm}[1]{\left\lVert#1\right\rVert}
%\providecommand{\norm}[1]{\lVert#1\rVert}
\providecommand{\mtx}[1]{\mathbf{#1}}
\providecommand{\mean}[1]{E\left[ #1 \right]}
\providecommand{\fourier}{\overset{\mathcal{F}}{ \rightleftharpoons}}
%\providecommand{\hilbert}{\overset{\mathcal{H}}{ \rightleftharpoons}}
\providecommand{\system}{\overset{\mathcal{H}}{ \longleftrightarrow}}
	%\newcommand{\solution}[2]{\textbf{Solution:}{#1}}
\newcommand{\solution}{\noindent \textbf{Solution: }}
\newcommand{\cosec}{\,\text{cosec}\,}
\providecommand{\dec}[2]{\ensuremath{\overset{#1}{\underset{#2}{\gtrless}}}}
\newcommand{\myvec}[1]{\ensuremath{\begin{pmatrix}#1\end{pmatrix}}}
\newcommand{\mydet}[1]{\ensuremath{\begin{vmatrix}#1\end{vmatrix}}}
\numberwithin{equation}{subsection}
\makeatletter
\@addtoreset{figure}{problem}
\makeatother
\let\StandardTheFigure\thefigure
\let\vec\mathbf
\renewcommand{\thefigure}{\theproblem}
\def\putbox#1#2#3{\makebox[0in][l]{\makebox[#1][l]{}\raisebox{\baselineskip}[0in][0in]{\raisebox{#2}[0in][0in]{#3}}}}
     \def\rightbox#1{\makebox[0in][r]{#1}}
     \def\centbox#1{\makebox[0in]{#1}}
     \def\topbox#1{\raisebox{-\baselineskip}[0in][0in]{#1}}
     \def\midbox#1{\raisebox{-0.5\baselineskip}[0in][0in]{#1}}
\vspace{3cm}
\title{CBSE Maths Questions 2007}
\author{Priyanka - EE21MTECH12002}
\maketitle
\newpage
\bigskip
\renewcommand{\thefigure}{\theenumi}
\renewcommand{\thetable}{\theenumi}
%
Get latex-tikz codes from 
%
\begin{lstlisting}
https://github.com/PeriPriyanka/cbsemathsquestions/2007_questions
\end{lstlisting}

\begin{enumerate}

\item (CBSE 2007-Question 2)
solve the values of x and y.
\begin{align}
x+\displaystyle\frac{6}{y}=6\\
3x-\displaystyle\frac{8}{y}=5
\end{align}
\solution Consider the equations 0.0.1 and 0.0.2 given in the problem statement.
\begin{align}
x+\displaystyle\frac{6}{y}=6\\
3x-\displaystyle\frac{8}{y}=5
\end{align}
The solution can be found by solving the above system of linear equations.\\ 
System of linear equations are defined as 
\begin{align}
\vec{AX=B}
\end{align}
From the equations 0.0.3 and 0.0.4, 
\begin{align}
\vec{A}= \myvec{1 &6\\3  & -8}\\
\medskip
\vec{X}= \myvec{ x\\ \displaystyle\frac{1}{y}}\\
\medskip
\vec{B}= \myvec{6\\5}  
\end{align} 
Substituting the values of $\vec{A}$, $\vec{X}$ and $\vec{B}$ in the equation 0.0.5
We get,
\begin{align}
\myvec{1&6\\3&-8} \myvec{x\\\displaystyle\frac{1}{y}}= \myvec{6\\5}
\end{align}
Considering the augmented matrix 
 \begin{align}
  \myvec{1&6&6\\3&-8&5}\\ 
 \xleftrightarrow[]{ R_2 \leftarrow R_2 - 3R_1}
  \myvec{1&6&6\\0&-26&-13}
 \end{align}
 \begin{align}
\myvec{1&6\\0&-26} \myvec{x\\\displaystyle\frac{1}{y}}= \myvec{6\\-13}\\
\medskip
x+\displaystyle\frac{6}{y}= 6\\
\medskip
\displaystyle\frac{-26}{y}=-13
\end{align}
By solving equations 0.0.14 we get,
\begin{align}
y= 2\
\end{align}
and by solving equation 0.0.13 we get ,
\begin{align}
x=3
\end{align}
Therefore, x=3 and y= 2 are solutions to the given equations 0.0.1 and 0.0.2
\bigskip
\item (CBSE 2007-Question 3)
solve the values of x and y
\begin{align}
\displaystyle\frac{x+1}{2}+\displaystyle\frac{y-1}{3}=8\\
\displaystyle\frac{x-1}{3}+\displaystyle\frac{y+1}{2}=9\end{align}

\solution Consider the equations 0.0.17 and 0.0.18 given in the problem statement.
\begin{align}
\displaystyle\frac{x+1}{2}+\displaystyle\frac{y-1}{3}=8\\
\displaystyle\frac{x-1}{3}+\displaystyle\frac{y+1}{2}=9
\end{align}
The above equations 0.0.19 and 0.0.20 can be rearranged as the following equations
\begin{align}
3x+2y=47\\
2x+3y=53
\end{align}
The solution can be found by solving the above system of linear equations.\\ 
System of linear equations are defined as 
\begin{align}
\vec{AX=B}
\end{align}
From the equations 0.0.21 and 0.0.22, 
\begin{align}
\vec{A}= \myvec{3 &2\\2 & 3}\\
\medskip
\vec{X}= \myvec{ x\ y}\\
\medskip
\vec{B}= \myvec{47\\53}  
\end{align} 
Substituting the values of $\vec{A}$, $\vec{X}$ and $\vec{B}$ in the equation 0.0.23
We get,
\begin{align}
\myvec{3&2\\2&3} \myvec{x\\y}= \myvec{47\\53}
\end{align}
Considering the augmented matrix 
 \begin{align}
 \myvec{3&2&47\\2&3&53}
 \\
\xleftrightarrow[]{ R_2 \leftarrow 3R_2 - 2R_1}
 \myvec{3&2&47\\0&5&65}
 \end{align}
 \begin{align}
\myvec{3&2\\0&5} \myvec{x\\y}= \myvec{47\\65}\\
\medskip
3x+2y= 47\\
\medskip
5y=65
\end{align}
By solving equations 0.0.32 we get,
\begin{align}
y= 13
\end{align}
and by solving equation 0.0.31 we get ,
\begin{align}
x=7
\end{align}
Therefore, x=7 and y= 13 are solutions to the given equations 0.0.17 and 0.0.18
\bigskip
\item (CBSE 2007-Question 21) Show that the points given below are vertices of an isosceles right angle triangle.
\begin{align}
\myvec{7\\10}\\ \myvec{-2\\5} \\ \myvec{3\\-4}
\end{align}

\solution Consider the given points as vectors,
\begin{align}
\vec{A}= \myvec{7\\10}\\
\vec{B}=\myvec{-2\\5}\\
\vec{C}=\myvec{3\\-4}
\end{align}
For a triangle to be an isosceles, any two sides of the triangle should be equal.
For finding a triangle to be isosceles and right angle, we consider,
\begin{align}
\vec{A-B}= \myvec{7\\10} - \myvec{-2\\5} = \myvec{9\\5}\\
\vec{B-C}= \myvec{-2\\5} - \myvec{3\\-4} = \myvec{-5 \\ 9}\\
\vec{C-A}=  \myvec{3\\-4} - \myvec{7\\10} = \myvec{-4\\ -14}
\end{align}
\begin{align}
(A-B)^T(B-C) = \myvec{9&5}\myvec{-5\\9} \\= -45+45 = 0\\
 (C-A)^T(A-B) = \myvec{-4&-14}\myvec{9\\5}\\ = -36-70 = -106\\
 (B-C)^T(C-A) = \myvec{-5&9}\myvec{-4\\-14}\\ = 20-126 = -106
\end{align}
From the equation 0.0.45 $\vec{A-B} \perp \vec{B-C}$, Therefore $\measuredangle{B} = 90\degree$\\
From the equations 0.0.47 and 0.0.49 $\measuredangle{CAB}=\measuredangle{BCA}$\\
Therefore, $\triangle{ABC}$ is an isosceles right angle triangle with sides $\vec{AB}=\vec{BC}$ and right angle at $\vec{B}$
\bigskip
\item (CBSE 2007-Question 22) In what ratio does the line x-y-2=0 divides the line segment joining $\myvec{3 &-1}$ and $\myvec{8&9} $?

\solution Consider the line x-y-2=0 divides the line segment $\myvec{3 \\1}$ and $\myvec{8\\9} $ in $k:1$ ratio.\\
$\vec{P}=\myvec{x& y} $ is point of intersection of two lines.\\ 
From the section formula we can write,
\begin{align}
\vec{P} = \myvec{x\\ y} = \displaystyle\frac{1}{k+1}\left[\myvec{3\\-1}+ k\myvec{8\\9}\right]\\
 = \myvec{\displaystyle\frac{3+3k}{k+1} \\\\ \displaystyle\frac{-1+9k}{k+1}}
\end{align}
The point $\vec{P}$ passes through the line x-y-2=0, therefore,
\begin{align}
\displaystyle\frac{3+3k}{k+1} - \displaystyle\frac{-1+9k}{k+1} -2 = 0\\
k=\displaystyle\frac{2}{3}
\end{align}
Therefore, the line x-y-2=0 divides the line segment $\myvec{3 \\1}$ and $\myvec{8\\9} $ in $2:3$ ratio.
\end{enumerate}
\end{document}