\documentclass[journal,12pt,twocolumn]{IEEEtran}

\usepackage{setspace}
\usepackage{gensymb}
\singlespacing
\usepackage[cmex10]{amsmath}

\usepackage{amsthm}

\usepackage{mathrsfs}
\usepackage{txfonts}
\usepackage{stfloats}
\usepackage{bm}
\usepackage{cite}
\usepackage{cases}
\usepackage{subfig}

\usepackage{longtable}
\usepackage{multirow}

\usepackage{enumitem}
\usepackage{mathtools}
\usepackage{steinmetz}
\usepackage{tikz}
\usepackage{circuitikz}
\usepackage{verbatim}
\usepackage{tfrupee}
\usepackage[breaklinks=true]{hyperref}
\usepackage{graphicx}
\usepackage{tkz-euclide}

\usetikzlibrary{calc,math}
\usepackage{listings}
    \usepackage{color}                                            %%
    \usepackage{array}                                            %%
    \usepackage{longtable}                                        %%
    \usepackage{calc}                                             %%
    \usepackage{multirow}                                         %%
    \usepackage{hhline}                                           %%
    \usepackage{ifthen}                                           %%
    \usepackage{lscape}     
\usepackage{multicol}
\usepackage{chngcntr}

\DeclareMathOperator*{\Res}{Res}

\renewcommand\thesection{\arabic{section}}
\renewcommand\thesubsection{\thesection.\arabic{subsection}}
\renewcommand\thesubsubsection{\thesubsection.\arabic{subsubsection}}

\renewcommand\thesectiondis{\arabic{section}}
\renewcommand\thesubsectiondis{\thesectiondis.\arabic{subsection}}
\renewcommand\thesubsubsectiondis{\thesubsectiondis.\arabic{subsubsection}}


\hyphenation{op-tical net-works semi-conduc-tor}
\def\inputGnumericTable{}                                 %%

\lstset{
%language=C,
frame=single, 
breaklines=true,
columns=fullflexible
}
\begin{document}


\newtheorem{theorem}{Theorem}[section]
\newtheorem{problem}{Problem}
\newtheorem{proposition}{Proposition}[section]
\newtheorem{lemma}{Lemma}[section]
\newtheorem{corollary}[theorem]{Corollary}
\newtheorem{example}{Example}[section]
\newtheorem{definition}[problem]{Definition}

\newcommand{\BEQA}{\begin{eqnarray}}
\newcommand{\EEQA}{\end{eqnarray}}
\newcommand{\define}{\stackrel{\triangle}{=}}
\bibliographystyle{IEEEtran}
\raggedbottom
\setlength{\parindent}{0pt}
\providecommand{\mbf}{\mathbf}
\providecommand{\pr}[1]{\ensuremath{\Pr\left(#1\right)}}
\providecommand{\qfunc}[1]{\ensuremath{Q\left(#1\right)}}
\providecommand{\sbrak}[1]{\ensuremath{{}\left[#1\right]}}
\providecommand{\lsbrak}[1]{\ensuremath{{}\left[#1\right.}}
\providecommand{\rsbrak}[1]{\ensuremath{{}\left.#1\right]}}
\providecommand{\brak}[1]{\ensuremath{\left(#1\right)}}
\providecommand{\lbrak}[1]{\ensuremath{\left(#1\right.}}
\providecommand{\rbrak}[1]{\ensuremath{\left.#1\right)}}
\providecommand{\cbrak}[1]{\ensuremath{\left\{#1\right\}}}
\providecommand{\lcbrak}[1]{\ensuremath{\left\{#1\right.}}
\providecommand{\rcbrak}[1]{\ensuremath{\left.#1\right\}}}
\theoremstyle{remark}
\newtheorem{rem}{Remark}
\newcommand{\sgn}{\mathop{\mathrm{sgn}}}
\providecommand{\abs}[1]{\left\vert#1\right\vert}
\providecommand{\res}[1]{\Res\displaylimits_{#1}} 
\providecommand{\norm}[1]{\left\lVert#1\right\rVert}
%\providecommand{\norm}[1]{\lVert#1\rVert}
\providecommand{\mtx}[1]{\mathbf{#1}}
\providecommand{\mean}[1]{E\left[ #1 \right]}
\providecommand{\fourier}{\overset{\mathcal{F}}{ \rightleftharpoons}}
%\providecommand{\hilbert}{\overset{\mathcal{H}}{ \rightleftharpoons}}
\providecommand{\system}{\overset{\mathcal{H}}{ \longleftrightarrow}}
	%\newcommand{\solution}[2]{\textbf{Solution:}{#1}}
\newcommand{\solution}{\noindent \textbf{Solution: }}
\newcommand{\cosec}{\,\text{cosec}\,}
\providecommand{\dec}[2]{\ensuremath{\overset{#1}{\underset{#2}{\gtrless}}}}
\newcommand{\myvec}[1]{\ensuremath{\begin{pmatrix}#1\end{pmatrix}}}
\newcommand{\mydet}[1]{\ensuremath{\begin{vmatrix}#1\end{vmatrix}}}
\numberwithin{equation}{subsection}
\makeatletter
\@addtoreset{figure}{problem}
\makeatother
\let\StandardTheFigure\thefigure
\let\vec\mathbf
\renewcommand{\thefigure}{\theproblem}
\def\putbox#1#2#3{\makebox[0in][l]{\makebox[#1][l]{}\raisebox{\baselineskip}[0in][0in]{\raisebox{#2}[0in][0in]{#3}}}}
     \def\rightbox#1{\makebox[0in][r]{#1}}
     \def\centbox#1{\makebox[0in]{#1}}
     \def\topbox#1{\raisebox{-\baselineskip}[0in][0in]{#1}}
     \def\midbox#1{\raisebox{-0.5\baselineskip}[0in][0in]{#1}}
\vspace{3cm}
\title{CBSE Maths Questions 2007}
\author{Priyanka - EE21MTECH12002}
\maketitle
\newpage
\bigskip
\renewcommand{\thefigure}{\theenumi}
\renewcommand{\thetable}{\theenumi}
Download all python codes from 
\begin{lstlisting}
https://github.com/PeriPriyanka/cbsemathsquestions/2007/12/matrices/codes/solutions
\end{lstlisting}
%
Get latex-tikz codes from 
%
\begin{lstlisting}
https://github.com/PeriPriyanka/cbsemathsquestions/2007/12/matrices/solutions
\end{lstlisting}

\begin{enumerate}

\item (CBSE 2007-Question 1)
If $\vec{A} = \myvec{2&-3\\3&4}$, show that $ \vec{A}^2-6\vec{A}+17\vec{I}=0$. Hence find $\vec{A}^{-1}$. 
\solution Consider the matrix  given in the problem statement.
\begin{align}
&\vec{A} = \myvec{2&-3\\3&4}&
\end{align}
Considering the characteristic equation:  
\begin{align} 
& \vert\vec{A}-\lambda\vec{I}\vert = 0  & \label{eq:0.0.2}
\end{align}
From \eqref{eq:0.0.2} we get,
\begin{align}
&\begin{vmatrix}
  2-\lambda & -3\\ 3& 4-\lambda 
\end{vmatrix}
=0 \\
&(2-\lambda)(4-\lambda)+9=0\\ 
&\lambda^2-6\lambda+17=0 \label{eq:0.0.5} 
\end{align}
From the Cayley-Hamilton theorem \eqref{eq:0.0.5} can be written as
\begin{align}
&\vec{A}^2-6\vec{A}+17\vec{I} = 0  \label{eq:0.0.6} 
\end{align} 
Multiplying with $\vec{A}^{-1}$ on both sides of equation \eqref{eq:0.0.6}
We get,
\begin{align}
  &\vec{A}-6\vec{I}+17\vec{A}^{-1} = 0  \\
  &\vec{A}^{-1} = \displaystyle\frac{6\vec{I}-\vec{A}}{17} \\
  &\vec{A}^{-1} =  \myvec{4/17 &  3/17 \\ -3/17 & 2/17 }
\end{align}
\item (CBSE 2007-Question 3) Using the properties of determinants, prove the following:
\begin{center}
$\begin{vmatrix}
a-b-c & 2a& 2a\\ 2b& b-c-a& 2b\\2c&2c&c-a-b 
\end{vmatrix}$
$= (a+b+c)^3$
\end{center}
\solution
 \begin{align}
  &\begin{vmatrix}
    a-b-c & 2a& 2a\\ 2b& b-c-a& 2b\\2c&2c&c-a-b 
    \end{vmatrix}\\
& \xleftrightarrow[]{ R_1 \leftarrow R_1 + R_2+R_3}  \nonumber \\
&\begin{vmatrix} a+b+c & a+b+c& a+b+c \\ 2b& b-c-a& 2b\\2c&2c&c-a-b  \end{vmatrix}\\
  &= (a+b+c)\begin{vmatrix} 1 & 1 & 1\\ 2b& b-c-a& 2b\\2c&2c&c-a-b  \end{vmatrix}\\
   & \xleftrightarrow[]{ C_2 \leftarrow C_2- C_1 , C_3 \leftarrow C_3 - C_1} \nonumber \\
   &(a+b+c)\begin{vmatrix} 1 & 0 & 0\\ 2b& a+b+c& 0\\2c&0&a+b+c \end{vmatrix}\\
 &= (a+b+c)(a+b+c)^2  \\
 &= (a+b+c)^3 
  \end{align}
\item (CBSE 2007-Question 19) Using matrices, solve the following system of equation:
\begin{align}
   & x+2y-3z=6  \label{eq:0.0.16}  \\
   &  3x+2y-2z=3  \label{eq:0.0.17}\\
   &  2x-y+z=2   \label{eq:0.0.18}
\end{align}
\solution
Consider the equations given in the problem statement.
The solution can be found by solving the above system of linear equations.\\ 
System of linear equations are defined as 
\begin{align}
\vec{Ax=B} \label{eq:0.0.19}
\end{align}
From the equations \eqref{eq:0.0.16} ,\eqref{eq:0.0.17} and \eqref{eq:0.0.18}, 
\begin{align}
&\vec{A}= \myvec{1 &2 &-3\\3 & 2& -2\\2&-1&1} \\
\medskip
&\vec{x}= \myvec{ x\\ y\\z}\\
\medskip
&\vec{B}= \myvec{6\\3\\2}
\end{align} 
Substituting the values of $\vec{A}$, $\vec{x}$ and $\vec{B}$ in the equation \eqref{eq:0.0.19}
We get,
\begin{align}
&\myvec{1&2&-3\\3&2&-2\\2&-1&1} \myvec{x\\y\\z}= \myvec{6\\3\\2}
\end{align}
Considering the augmented matrix 
 \begin{align}
& \myvec{1&2&-3&6\\3&2&-2&3\\2&-1&1&2}
 \\
&\xleftrightarrow[]{ R_2 \leftarrow R_2 - 3R_1 , R_3 \leftarrow R_3 - 2R_1}
\myvec{1&2&-3&6\\0&-4&7&-15\\0&-5&7&-10}\\
 &\xleftrightarrow[]{ R_3 \leftarrow 4R_3 - 5R_2}
 \myvec{1&2&-3&6\\0&-4&7&-15\\0&0&-7&35}\\
 &\xleftrightarrow[]{ R_3 \leftarrow R_3/-7}
 \myvec{1&2&-3&6\\0&-4&7&-15\\0&0&1&-5}\\
 &\xleftrightarrow[]{ R_2 \leftarrow R_2-7R_3}
 \myvec{1&2&-3&6\\0&-4&0&20\\0&0&1&-5}\\
 &\xleftrightarrow[]{ R_2 \leftarrow R_2/-4}
 \myvec{1&2&-3&6\\0&1&0&-5\\0&0&1&-5}\\
 &\xleftrightarrow[]{ R_1 \leftarrow R_1-2R_2,R_1 \leftarrow R_1+3R_2}
 \myvec{1&0&-0&1\\0&1&0&-5\\0&0&1&-5}
 \end{align}
 \begin{align}
&\myvec{1&0&0\\0&1&0\\0&0&1} \myvec{x\\y\\z}= \myvec{1\\-5\\-5}& \label{eq:0.0.30}
\medskip
\end{align}
By solving equation \eqref{eq:0.0.30} we get,
\begin{align}
&x=1\\
&y= -5\\
&z=-5
\end{align}
Therefore, x=1, y= -5 and z=-5 are solutions to the given equations. 
\bigskip
\item (CBSE 2007-Question 24) Find the projection of $\overrightarrow{\vec{b}}+\overrightarrow{\vec{c}}$ on $\overrightarrow{\vec{a}}$ where $\overrightarrow{\vec{a}}=2\hat{\vec{i}}-2\hat{\vec{j}}+\hat{\vec{k}} , \overrightarrow{\vec{b}}=\hat{\vec{i}}+2\hat{\vec{j}}-2\hat{\vec{k}}$ and $ \overrightarrow{\vec{c}}=2\hat{\vec{i}}-\hat{\vec{j}}+4\hat{\vec{k}}$\\
\solution Consider the given vectors,
\begin{align}
&\vec{A}= \myvec{2\\-2\\1}\\
&\vec{B}=\myvec{1\\2\\-2}\\
&\vec{C}=\myvec{2\\-1\\4}
\end{align}
Projection of $\vec{(B+C)}$ on $\vec{A}$ is given by 
\begin{align}
&\displaystyle\frac{\vec{(B+C)}^\text T\vec{A}}{\|\vec{A}\|} \label{eq:0.0.38}
\end{align}
By substituting $\vec{A}$, $\vec{B}$ and $\vec{C} $ in \eqref{eq:0.0.38} we get, 
\begin{align}
  &\displaystyle\frac{\vec{(B+C)}^\text T\vec{A}}{\|\vec{A}\|}
&=\displaystyle\frac{\myvec{3&1&2}.\myvec{2\\-2\\1}}{\sqrt{4+4+1}} 
&=2 
\end{align}
\item (CBSE 2007-Question 25) Find the value of $\lambda$ which makes the vectors $\overrightarrow{\vec{a}}$,$\overrightarrow{\vec{b}}$ and $\overrightarrow{\vec{c}}$ coplanar, where $ \overrightarrow{\vec{a}}=2\hat{\vec{i}}-\hat{\vec{j}}+\hat{\vec{k}}$,$ \overrightarrow{\vec{b}}=\hat{\vec{i}}+2\hat{\vec{j}}-3\hat{\vec{k}}$ and $ \overrightarrow{\vec{c}}=3\hat{\vec{i}}-\lambda\hat{\vec{j}}+5\hat{\vec{k}}$\\
\solution Consider the given vectors,
\begin{align}
&\vec{A}= \myvec{2\\-1\\1}\\
&\vec{B}=\myvec{1\\2\\-3}\\
&\vec{C}=\myvec{3\\-\lambda\\5}
\end{align}
For the vectors $\vec{A}$, $\vec{B}$ and $\vec{C}$ to be coplanlar, the three vectors are linearly dependent. Therfore,
\begin{align}
  &\begin{vmatrix}
    2 & -1& 1\\ 1& 2& -3\\3&-\lambda&5 
    \end{vmatrix} = 0 \\
& = 2(10-3\lambda)+1(5+9)+1(-\lambda -6)=0\\
& \lambda=4
\end{align}
\item (CBSE 2007-Question 31) Find the equation of the plane which is perpendicular to the plane $5x + 3y + 6 z + 8 = 0$ and which contains the line of intersection of the planes $x + 2y + 3z - 4 = 0 $ and $2x + y - z + 5 = 0$.\\ 
\solution consider the given planes as
\begin{align}
&\vec{A}^\text T\vec{x}=\text c_1\nonumber \\&=\myvec{5&3&6}\myvec{x\\y\\z}=-8 \label{eq:0.0.46}\\ 
&\vec{B}^\text T\vec{x}=\text c_2\nonumber \\&=\myvec{1&2&3}\myvec{x\\y\\z}=4\\
  &\vec{C}^\text T\vec{x}=\text c_3\nonumber \\&=\myvec{2&1&-1}\myvec{x\\y\\z}=-5
\end{align}
Plane $\perp$ to $5x + 3y + 6 z + 8 = 0$ is given by,
\begin{align}
&(\vec{B}+\text k\vec{C})\vec{x}= \text c\\
&\left(\myvec{1\\2\\3}+\text k\myvec{2\\1\\-1}\right)^\text T.\myvec{x\\y\\z} =4-5\text k \label{eq:0.0.50} 
\end{align}
The plane in equation \eqref{eq:0.0.50} is $\perp$ to plane in equation \eqref{eq:0.0.46}. Therefore,
\begin{align}  
&\vec{A}^\text T. \left(\vec{B}+\text k\vec{C}\right) =0\\
&\myvec{5&3&6}.\left[\myvec{1\\2\\3}+\text k\myvec{2\\1\\-1}\right]=0\\
&\text k= \displaystyle\frac{-29}{7}
\end{align}
The required plane is \begin{align} \myvec{51&15&-50}\myvec{x\\y\\z}=-173 \end{align}
\end{enumerate}
\end{document}